%%%%%%%%%%%%%%%%%%%%%%%%%%%%%%%%%%%%%%%%%
% Programming/Coding Assignment
% LaTeX Template
%
% This template has been downloaded from:
% http://www.latextemplates.com
%
% Original author:
% Ted Pavlic (http://www.tedpavlic.com)
%
% Note:
% The \lipsum[#] commands throughout this template generate dummy text
% to fill the template out. These commands should all be removed when 
% writing assignment content.
%
% This template uses a Perl script as an example snippet of code, most other
% languages are also usable. Configure them in the "CODE INCLUSION 
% CONFIGURATION" section.
%
%%%%%%%%%%%%%%%%%%%%%%%%%%%%%%%%%%%%%%%%%

%----------------------------------------------------------------------------------------
%	PACKAGES AND OTHER DOCUMENT CONFIGURATIONS
%----------------------------------------------------------------------------------------

\documentclass{article}

\usepackage{fancyhdr} % Required for custom headers
\usepackage{lastpage} % Required to determine the last page for the footer
\usepackage{extramarks} % Required for headers and footers
\usepackage[usenames,dvipsnames]{color} % Required for custom colors
\usepackage{graphicx} % Required to insert images
\usepackage{listings} % Required for insertion of code
\usepackage{courier} % Required for the courier font
\usepackage{multirow}
\usepackage{hyperref}
\usepackage{amsmath}
\usepackage{amssymb}

% Margins
\topmargin=-0.45in
\evensidemargin=0in
\oddsidemargin=0in
\textwidth=6.5in
\textheight=9.0in
\headsep=0.25in

\linespread{1.1} % Line spacing

%----------------------------------------------------------------------------------------
%	CODE INCLUSION CONFIGURATION
%----------------------------------------------------------------------------------------

\definecolor{MyDarkGreen}{rgb}{0.0,0.4,0.0} % This is the color used for comments
\lstloadlanguages{c} % Load Perl syntax for listings, for a list of other languages supported see: ftp://ftp.tex.ac.uk/tex-archive/macros/latex/contrib/listings/listings.pdf
\lstset{language=[sharp]c, % Use Perl in this example
        frame=single, % Single frame around code
        basicstyle=\small\ttfamily, % Use small true type font
        keywordstyle=[1]\color{Blue}\bf, % Perl functions bold and blue
        keywordstyle=[2]\color{Purple}, % Perl function arguments purple
        keywordstyle=[3]\color{Blue}\underbar, % Custom functions underlined and blue
        identifierstyle=, % Nothing special about identifiers                                         
        commentstyle=\usefont{T1}{pcr}{m}{sl}\color{MyDarkGreen}\small, % Comments small dark green courier font
        stringstyle=\color{Purple}, % Strings are purple
        showstringspaces=false, % Don't put marks in string spaces
        tabsize=5, % 5 spaces per tab
        %
        % Put standard Perl functions not included in the default language here
        morekeywords={rand},
        %
        % Put Perl function parameters here
        morekeywords=[2]{on, off, interp},
        %
        % Put user defined functions here
        morekeywords=[3]{test},
       	%
        morecomment=[l][\color{Blue}]{...}, % Line continuation (...) like blue comment
        numbers=left, % Line numbers on left
        firstnumber=1, % Line numbers start with line 1
        numberstyle=\tiny\color{Blue}, % Line numbers are blue and small
        stepnumber=5 % Line numbers go in steps of 5
}

\newcommand{\horrule}[1]{\rule{\linewidth}{#1}}

% Creates a new command to include a perl script, the first parameter is the filename of the script (without .pl), the second parameter is the caption
\newcommand{\perlscript}[2]{
\begin{itemize}
\item[]\lstinputlisting[caption=#2,label=#1]{#1.cs}
\end{itemize}
}

\begin{document}

\begin{tabular}{l l}
\multirow{6}{*}{\includegraphics[width=5cm]{../../recursos/logo.jpg}}
 & Universidad Francisco Marroquin \\
 & Facultad de Ciencias Economicas \\
 & Computer Science \\
 & Algoritmos y Complejidad \\
 & Prof. Ernesto Rodriguez - \href{mailto:erodriguez@unis.edu.gt}{erodriguez@unis.edu.gt} \\
 & Aux. Juan Roberto Alvarado
\end{tabular}
\\\\\\

\begin{center}
        \horrule{0.5pt}
        \huge{Proyecto Final -- Code Crawler} \\
        \large{Fecha de entrega: 22 de Noviembre, 2018 - 11:59pm} \\
        \horrule{1pt}
\end{center}

\emph{Instrucciones: Resolver cada uno de los ejercicios siguiendo sus respectivas
instrucciones. El trabajo debe ser entregado a traves de Github, en su repositorio del curso, colocado en una
carpeta llamada "Proyecto Final". Al menos que la pregunta indique diferente, todas las
respuestas a preguntas escritas deben presentarse en un documento formato pdf, el cual
haya sido generado mediante Latex. }
\section{Descripci\'on General}
El objetivo de este proyecto es aplicar algunos de los algoritmos aprendidos en clase en
una situaci\'on de la vida real. Su tarea es implementar un \emph{crawler} que permita
indexar y buscar objetos en codigo {\bf Javascript} (es6), similar al programa \href{https://www.haskell.org/hoogle/}{Hoogle}.
\\\\
Para ello, su programa debe inspeccionar codigo fuente Javascript y utilizar las expressiones
{\bf require} para ubicar los demas archivos y construir un indice de objetos (metodos, constantes,
funciones, clases, etc.). Aparte el programa debe prover una interfaz (grafica, web, linea de comandos,
rest, etc.) para buscar objetos. Usted debe utilizar su creatividad para dise\~nar la interfaz de
tal forma que permita filtrar los resultados de formas utiles. Por ejemplo, hacer una busqueda solamente
de clases que hereden de la clase ``Foo''. Debe existir un {\bf arbol abarcador} el cual se origina en
el archivo que se haya dado como parametro al programa y este arbol abarque todos los archivos que se
puedan localizar desde el origen.

\section{Evaluaci\'on}
El proceso de evaluaci\'on consistira de los siguientes pasos:
\begin{enumerate}
        \item{Se le proveera a su programa un archivo Javascript para que construya el indice.
        Su programa debe buscar enunciados {\bf require} para descubrir los archivos restantes.}
        \item{Luego de terminar el proceso de indexado, se probara el buscador mediante algunas
        busquedas de prueba. Por ejemplo:
        \begin{itemize}
                \item{Buscar las clases con combre similar a ``Foo''}
                \item{Buscar los metodos que acepten 2 parametros}
                \item{Buscar las clases que heredan de ``Bar''}
        \end{itemize}}
        \item{Por ultimo, se verificara que para cada resultado de la busqueda, tenga
        un camino dentro del {\bf arbol abarcador} del indice de archivos}
\end{enumerate}

Tambien se llevara a cabo una competencia contra los demas compa\~neros. Se utilizara un proyecto
Javascript particular, el cual sera indexado y se realizaran 10 busquedas (no se revelaran exactamente
cuales) en el codigo de dicho proyecto. Los trabajos que mejor localizan los objetos buscados seran
los ganadores de la competencia. Se premiaran de la siguiente manera:
\begin{enumerate}
        \item{{\bf Primer lugar:} 5 puntos extra en el curso}
        \item{{\bf Segundo lugar}: 3 puntos extra en el curso}
        \item{{\bf Tercer lugar:} 1 punto extra en el curso}
\end{enumerate}

\section{Recomendaciones}

Para facilitar el procesamiento del codigo fuente, se recomienda procesar los archivos Javascript
con un {\bf lexer} ya existente. Algunos de los existentes son:
\begin{enumerate}
        \item{Pygments (Python): \url{https://bitbucket.org/birkenfeld/pygments-main/src/default/pygments/lexers/javascript.py}}
        \item{v8 (C++): \url{https://v8.dev/}}
        \item{Language Javascript (Haskell): \url{https://hackage.haskell.org/package/language-javascript}}
        \item{Nashorn (Java): \url{https://openjdk.java.net/projects/nashorn/}}
        \item{Rhino (Java): \url{https://github.com/mozilla/rhino}}
\end{enumerate}

\section{Entrega}
El trabajo se debe entregar a traves de {\bf Github}. Por favor entregar un proyecto de calidad
profesional tal como lo espera la industria y academia de tecnologia esto incluye:
\begin{itemize}
        \item{Asegurarse que el programa pueda ser compilado en la plataforma para la cual
        fue escrito. Idealmente si puede compilarse en cualquier plataforma}
        \item{Instrucciones detallando como compilar y utilizar el programa}
        \item{Utilizar un sistema de compilaci\'on: Make, CMake, Ant, Maven, Pip, etc.}
        \item{Si la plataforma lo permite, hacerlo un paquete. Ejemplo: NPM o Pip}
        \item{Detallar los requisitos del sistema para compilar y ejecutar el programa}
        \item{Ejemplos y pruebas unitarias}
\end{itemize}

\end{document}