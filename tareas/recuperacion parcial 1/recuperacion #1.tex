%%%%%%%%%%%%%%%%%%%%%%%%%%%%%%%%%%%%%%%%%
% Programming/Coding Assignment
% LaTeX Template
%
% This template has been downloaded from:
% http://www.latextemplates.com
%
% Original author:
% Ted Pavlic (http://www.tedpavlic.com)
%
% Note:
% The \lipsum[#] commands throughout this template generate dummy text
% to fill the template out. These commands should all be removed when 
% writing assignment content.
%
% This template uses a Perl script as an example snippet of code, most other
% languages are also usable. Configure them in the "CODE INCLUSION 
% CONFIGURATION" section.
%
%%%%%%%%%%%%%%%%%%%%%%%%%%%%%%%%%%%%%%%%%

%----------------------------------------------------------------------------------------
%	PACKAGES AND OTHER DOCUMENT CONFIGURATIONS
%----------------------------------------------------------------------------------------

\documentclass{article}

\usepackage{fancyhdr} % Required for custom headers
\usepackage{lastpage} % Required to determine the last page for the footer
\usepackage{extramarks} % Required for headers and footers
\usepackage[usenames,dvipsnames]{color} % Required for custom colors
\usepackage{graphicx} % Required to insert images
\usepackage{listings} % Required for insertion of code
\usepackage{courier} % Required for the courier font
\usepackage{multirow}
\usepackage{hyperref}
\usepackage{amsmath}
\usepackage{amssymb}

% Margins
\topmargin=-0.45in
\evensidemargin=0in
\oddsidemargin=0in
\textwidth=6.5in
\textheight=9.0in
\headsep=0.25in

\linespread{1.1} % Line spacing

%----------------------------------------------------------------------------------------
%	CODE INCLUSION CONFIGURATION
%----------------------------------------------------------------------------------------

\definecolor{MyDarkGreen}{rgb}{0.0,0.4,0.0} % This is the color used for comments
\lstloadlanguages{c} % Load Perl syntax for listings, for a list of other languages supported see: ftp://ftp.tex.ac.uk/tex-archive/macros/latex/contrib/listings/listings.pdf
\lstset{language=[sharp]c, % Use Perl in this example
        frame=single, % Single frame around code
        basicstyle=\small\ttfamily, % Use small true type font
        keywordstyle=[1]\color{Blue}\bf, % Perl functions bold and blue
        keywordstyle=[2]\color{Purple}, % Perl function arguments purple
        keywordstyle=[3]\color{Blue}\underbar, % Custom functions underlined and blue
        identifierstyle=, % Nothing special about identifiers                                         
        commentstyle=\usefont{T1}{pcr}{m}{sl}\color{MyDarkGreen}\small, % Comments small dark green courier font
        stringstyle=\color{Purple}, % Strings are purple
        showstringspaces=false, % Don't put marks in string spaces
        tabsize=5, % 5 spaces per tab
        %
        % Put standard Perl functions not included in the default language here
        morekeywords={rand},
        %
        % Put Perl function parameters here
        morekeywords=[2]{on, off, interp},
        %
        % Put user defined functions here
        morekeywords=[3]{test},
       	%
        morecomment=[l][\color{Blue}]{...}, % Line continuation (...) like blue comment
        numbers=left, % Line numbers on left
        firstnumber=1, % Line numbers start with line 1
        numberstyle=\tiny\color{Blue}, % Line numbers are blue and small
        stepnumber=5 % Line numbers go in steps of 5
}

\newcommand{\horrule}[1]{\rule{\linewidth}{#1}}

% Creates a new command to include a perl script, the first parameter is the filename of the script (without .pl), the second parameter is the caption
\newcommand{\perlscript}[2]{
\begin{itemize}
\item[]\lstinputlisting[caption=#2,label=#1]{#1.cs}
\end{itemize}
}

\begin{document}

\begin{tabular}{l l}
\multirow{6}{*}{\includegraphics[width=5cm]{../../recursos/logo.jpg}}
 & Universidad Francisco Marroquin \\
 & Facultad de Ciencias Economicas \\
 & Computer Science \\
 & Algoritmos y Complejidad \\
 & Prof. Ernesto Rodriguez - \href{mailto:erodriguez@unis.edu.gt}{erodriguez@unis.edu.gt} \\
 & Aux. Juan Roberto Alvarado
\end{tabular}
\\\\\\

\begin{center}
        \horrule{0.5pt}
        \huge{Recuperaci\'on Parcial \#1} \\
        \large{Fecha de entrega: 04 de Octubre, 2018 - 11:59pm} \\
        \horrule{1pt}
\end{center}

\emph{Instrucciones: Resolver cada uno de los ejercicios siguiendo sus respectivas
instrucciones. El trabajo debe ser entregado a traves de Github, en su repositorio del curso, colocado en una
carpeta llamada "Recuperaci\'on 1". Al menos que la pregunta indique diferente, todas las
respuestas a preguntas escritas deben presentarse en un documento formato pdf, el cual
haya sido generado mediante Latex. }\\\\

El objetivo de este trabajo es implementar el algoritmo de ordenamiento \emph{merge sort}
utilizando un framework de computaci\'on distribuida. Tiene la libertad de escoger
el framework y lenguaje, a continuaci\'on se presentan algunas sugerencias:
\begin{itemize}
        \item{\href{https://haskell-distributed.github.io/}{Cloud Haskell} (Haskell)}
        \item{\href{https://dotnet.github.io/orleans/}{Orleans} (.Net)}
        \item{\href{https://pgiri.github.io/dispy/}{Dispy} (Python)}
        \item{\href{https://www.erlang.org/}{Erlang} (Erlang)}
        \item{\href{https://www.open-mpi.org/software/ompi/v1.10/}{MPI} (c/c++)}
        \item{\href{https://akka.io/}{Akka} (JVM)}
\end{itemize}

\section*{Descripci\'on}
Su programa debe consumir una lista de oraciones en un documento de texto. Cada oraci\'on
ocurrira en una linea diferente. El objetivo es distribuir la ejecuci\'on de merge
sort en varios procesos o computadroas debido a que los frameworks de computaci\'on
distribuidas son transparentes ante el interntet.
\\\\
Al final de la ejecuci\'on de su programa, se debe producir una lista que tenga
todas las oraciones de la lista original ordenadas alfabeticamente por el codigo
ascii. Puede asumir que solo se utilizaran letras (mayusculas y minusculas), numeros
y espacios (no habran signos de puntuaci\'on).

\section*{Evaluaci\'on}
Una implementaci\'on funcional de este algoritmo con ejecuci\'on distribuida
sera merecedora de un (30\%) adicional en su nota del examen parcial. Adicionalmente,
los siguientes criterios seran evaluados para puntos extra:
\begin{itemize}
        \item{Lectura en paralelo de la lista de oraciones}
        \item{Escritura en paralelo de la lista de oraciones}
        \item{Criterios arbitrarios de ordenamiento}
        \item{Utilizaci\'on eficiente de nodos. En otras palabras,
        distribuir la carga optimamente seg\'un el numero de nodos
        que se disponga para el procesamiento.}
        \item{Interfaz web o grafica que facilite la utilizaci\'on
        de su algoritmo}
        \item{Lectura distribuida de archivos. En otras palabras que
        los nodos puedan leer parte de la data localmente y retornar
        un resultado consolidado al nodo principal.}
\end{itemize}

\end{document}