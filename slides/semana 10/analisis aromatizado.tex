\documentclass{beamer}
\usepackage[utf8]{inputenc}
\usepackage{hyperref}
\usepackage{multicol}
\usepackage{hyperref}
\usepackage{amsmath}
\usepackage[english]{babel}
\usepackage{algorithm}
\usepackage[noend]{algpseudocode}

\inputencoding{utf8}

\mode<presentation> {
    \usetheme{Madrid}
}

\usepackage{graphicx}
\usepackage{booktabs}

\title[Codiciosos]{Analisis Aromatizado}
\author{Ernesto Rodriguez - Juan Roberto Alvaro Saravia}
\institute{
    Universidad Francisco Marroquin \\
    \medskip \textit{ernestorodriguez@ufm.edu - juanalvarado@ufm.edu}
}

\date[\today]{}

\begin{document}

\begin{frame}
\titlepage
\end{frame}

\begin{frame}
\frametitle{Metodo de Conteo}
\begin{itemize}
    \item{A cada operaci\'on se le asigna un {\bf costo}}
    \item{Este costo se conoce como {\bf costo aromatizado}}
    \item{Si una operaci\'on tiene un costo menor que su
    {\bf costo aromatizado}, la diferencia se convierte
    en {\bf credito}.}
    \item{El {\bf credito} se puede utilizar para compensar
    operaciones con costo m\'as alto que su {\bf costo aromatizado}.}
    \item{El {\bf credito} debe ser positivo en todo momento.}
\end{itemize}
\end{frame}

\begin{frame}
\frametitle{Ejemplo: Pila}
\begin{itemize}
    \item{La pila se encuentra en un estado inicial antes
    de la ejecuci\'on del porgrama (pila vacia)}
    \item{El estado de la pila se modificara mediante
    una serie de operaciones.}
    \item{Cada operaci\'on debe dejar suficiente credito
    para poder aplicar la operaci\'on inversa (si es que la hay).}
    \item{En otras palabras, el credito disponible debe ser suficiente
    para que el comportamiento de la pila con credito sea consistente
    con el comportamiento esperado.}
\end{itemize}
\end{frame}

\begin{frame}
\frametitle{Ejemplo: Una Pila}
{\bf Previamente:}
\begin{tabular}{l l}
    Push & 1 \\
    Pop & 1 \\
    Multipop & $min(s,k)$
\end{tabular}
{\bf Ahora:}
\begin{tabular}{l l}
    Push & 2 \\
    Pop & 0 \\
    Multipop & 0
\end{tabular}
\end{frame}

\begin{frame}
\frametitle{Ejemplo: Una Pila}
{\bf Intuici\'on:}
\begin{itemize}
    \item{Cada vez que hacemos push, adicional al elemento que fue empujado
    se empuja una unidad de credito.}
    \item{Cuando se desea hacer pop, se consume la unidad de credito que fue
    dejada por la operacion push.}
    \item{De esta manera, siempre se dispone de credito para la operaci\'on
    pop sin importar el estado de la pila.}
    \item{Debido a que solo se ``cuentan'' las operaci\'ones push, el costo
    aromatizado es $\mathcal{O}(n)$ -- proporcional al numero de push.}
\end{itemize}
{\bf Pregunta: } ¿Que sucede si la pila no esta vacia al principio?
\end{frame}

\begin{frame}
\frametitle{Ejemplo: Contador Binario}
\begin{itemize}
    \item{¿Cual es la operaci\'on que tiene un costo al incrementar un contador binario?}
    \item{¿Que operaci\'on invierte el cambio hecho anteriormente?}
    \item{¿Como podemos acreditar la 1era operaci\'on para que se puede llevar a cabo la segunda?}
\end{itemize}
\end{frame}

\begin{frame}
\frametitle{Ejemplo: Contador Binario}
\begin{itemize}
    \item{Cambiar un bit es la operaci\'on costosa en este ejemplo}
    \item{{\bf Idea:} Cuano un bit se transforma de 0 a 1, se paga el costo.}
    \item{Adicionalmente, se abona credito para poder regresar dicho bit a cero.}
    \item{Cuando el bit se transforma a cero, se utiliza el credito abonado previamente.}
\end{itemize}
\end{frame}

\begin{frame}
\frametitle{Ejemplo: Contador Binario}
\begin{itemize}
    \item{Cada operaci\'on, a lo sumo, convierte un bit de 0 a 1}
    \item{Por lo tanto, cada operaci\'on tiene un costo de 2.}
    \item{Los bits que pasan de 1 a 0 utilizan el credito disponible en cada
    bit para lograrlo.}
    \item{¿Cual es el costo total?}
\end{itemize}
\end{frame}

\begin{frame}
    \frametitle{Ejemplo: Contador Binario}
    \begin{itemize}
        \item{Cada operaci\'on, a lo sumo, convierte un bit de 0 a 1}
        \item{Por lo tanto, cada operaci\'on tiene un costo de 2.}
        \item{Los bits que pasan de 1 a 0 utilizan el credito disponible en cada
        bit para lograrlo.}
        \item{¿Cual es el costo total?
        \begin{itemize}
            \item{$\mathcal{O}(n)$}
        \end{itemize}
        }
    \end{itemize}
\end{frame}

\begin{frame}
    \frametitle{Ejemplo: Contador Binario}
    \begin{itemize}
        \item{¿Que sucede si el contador no inica en cero?}
        \item{¿Que sucede si se quiere decrementar el contador?}
    \end{itemize}
\end{frame}

\begin{frame}
\frametitle{Metodo potencial}
\begin{itemize}
    \item{El {\bf credito} acumulado se representa como un potencial,
    no como credito asociado a un objeto.}
    \item{El potencial puede ser liberado para financiar trabajo
    futuro.}
    \item{Se asocia un {\bf potencial} a la estructura de datos,
    no a valores que se encuentran en ella.
    \begin{itemize}
        \item{Se define una funci\'on $\Phi(D_i)$ la cual determina
        el potencial almacenado en la estructura $D_i$}
        \item{El indicie $i$ de la estructura $D_i$ indica el
        numero de operaciones que han sido aplicados a dicha
        estructura.}
    \end{itemize}
    }
\end{itemize}

\end{frame}

\begin{frame}
\frametitle{Metodo Potencial}
\begin{itemize}
    \item{Sea $c_i$ el costo de la operaci\'on numero $i$}
    \item{El {\bf costo aromatizado} $\hat c_i=c_i+\Phi(D_i)-\Phi(D_{i-1})$ de
    la operaci\'on $i$}
    \item{El {\bf costo aromatizado} de $n$ operaci\'ones por lo tanto es:
    \begin{itemize}
        \item{$\sum_{i=1}^{n} \hat c_i = \sum_{i=1}^{n}(c_i+\Phi(D_i) + \Phi(D_{i-1})=
        \sum_{i=1}^n \hat c_i + \Phi(D_n) - \Phi(D_0)$}
    \end{itemize}
    }
    \item{Si $\Phi(D_n)\geq \Phi(D_0)$, la suma de costos aromatizados es un limite
    superior al costo actual.}
    \item{Por lo general, se desconoce $n$, lo que significa que $\Phi(D_i)\geq \Phi(D_0)$
    para todo $i$}
\end{itemize}
\end{frame}

\begin{frame}
\frametitle{Ejemplo: Pila}
\begin{itemize}
    \item{$\Phi(D_i)$ es el numero de objetos en la pila.
        \begin{itemize}
            \item{El valor de $\Phi(D_0)$ es $0$}
            \item{Por lo tanto $\forall i>0.\ \Phi(D_i)>\Phi(D_0)$}
        \end{itemize}
    \item{¿Cual es el costo de la operaci\'on push?}
    \item{¿Cual es el costo de la operaci\'on pop?}
    }
\end{itemize}
\end{frame}

\end{document}